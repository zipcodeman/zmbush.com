%       File: cheat1.tex
%     Created: Sun Feb 26 05:00 PM 2012 P
% Last Change: Sun Feb 26 05:00 PM 2012 P
%
\documentclass[letterpaper]{article}
\usepackage[margin=0.5in]{geometry}
\usepackage{multicol,amsmath}
\begin{document}
\begin{multicols}{3}
\subsubsection*{Master Theorem}
For the recurrence $T(n) = aT(n/b) + O(n^d)$, the solution is as follows:
\begin{equation*}
T(n) = 
\begin{cases}
O(n^d),&\text{if } d > \log_b a\\
O(n^d \log n),&\text{if } d = \log_b a\\
O(n^{\log_b a}),&\text{if } d < \log_b a
\end{cases}
\end{equation*}
\subsubsection*{Roots of unity}
Are of the form $\omega^{0}, \omega^{1} \cdots, \omega^{n-1}$, where $\omega$ is
equal to $e^{2 \pi \i / n}$

\subsubsection*{Fast Fourier Transform}
The algorithm for the FFT is as follows:
\begin{verbatim}
FFT(a, w):
  if w == 1:
    return a
  (s0, s1, ..., sn/2-1) 
   = FFT((A0, A2, ..., An-2), w^2)
  (s'0, s'1, ..., s'n/2-1) 
   = FFT((A1, A3, ..., An-1), w^2)
  for j = 0 to (n/2)-1:
    rj = sj + (w^j)(s'j)
    rj+n/2 = sj - (w^j)(s'j)
  return (r0, r1, ..., rn-1)
\end{verbatim}
To perform the inverse operation, simply run the algorithm again with:
$\omega^{-1}$, and divide the result by $n$.
\end{multicols}
\subsubsection*{Edge Types}
\subsubsection*{Searches}
\subsubsection*{Bellman-Ford}
\subsubsection*{Djikstra's}
\end{document}


