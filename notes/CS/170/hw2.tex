% Search for all the places that say "PUT SOMETHING HERE".


\documentclass[11pt]{article}
\usepackage{amsmath,textcomp,amssymb,geometry,graphicx,amsthm}

\def\Name{Zachary Bush}  % Your name
\def\Sec{Th 4-5; Starfield}  % Your discussion section
\def\Login{cs170-nx} % Your login

\title{CS170--Spring 2012 --- Solutions to Homework 1}
\author{\Name, section \Sec, \texttt{\Login}}
\markboth{CS170--Spring 2012 Homework 2 \Name, section \Sec}{CS170--Spring 2012
Homework 2 \Name, section \Sec, \texttt{\Login}}
\pagestyle{myheadings}

\begin{document}
\maketitle

\begin{enumerate}
\item 

Collaborators: Mary Stufflebeam, Kelsey Theriault, Jacob Newman
\begin{enumerate}

\item
\begin{equation*}
T(n) = 3T(n/2) + O(n)
\end{equation*}
\begin{eqnarray*}
T(n) &=& 3[3T(n/4) + cn/2)] + cn = 9T(n/4) + \frac{5}{2}cn \\
&=& 9[3T(n/8) + cn/4] + \frac52cn = 27T(n/8) + \frac{19}4cn\\
&=& 27[3T(n/16) + cn/8] + \frac{19}4cn = 81T(n/16) + \frac{65}8cn\\
&=& 3^kT(n/2^k) + (\frac{3^k}{2^k} + \frac{3^{k-1}2}{2^k} + \dots)
\end{eqnarray*}
In this case, $k$ should be $\log_2 n$, which gives us:
\begin{equation*}
T(n) = 3^{\log_2 n}T(1) + \sum_{i=0}^{k}\frac{3^{i}2^{k-i}}{2^k}
\end{equation*}

The first term dominates here, so the answer is:
\begin{equation*}
O(n^{\log_2 3})
\end{equation*}

\item 
\begin{eqnarray*}
T(n) &=& T(n − 1) + O(1)\\
T(n) &\le& T(n-1) + c\\
&=& T(n-2) + 2c \\
&=& T(n-3) + 3c \\
&=& T(n-k) + kc \\
\end{eqnarray*}
In this case, our k is equal to n, so our value is:
\begin{equation*}
O(n)
\end{equation*}
\end{enumerate}

\newpage

\item 
Collaborators: Mary Stufflebeam, Kelsey Theriault
\begin{enumerate}
\item 
\begin{equation*}
T(n) = 5T\left(\frac{n}{2}\right) + O(n) = O(n^{\log_2 5})
\end{equation*}
\item
\begin{equation*}
T(n) = 2T\left(n - 1\right) + O(1) = O(2^n)
\end{equation*}
\item
\begin{equation*}
T(n) = 9T\left(\frac{n}{3}\right) + O(n^2) = O(n^2\log n)
\end{equation*}
\end{enumerate}

The best algorithm of these options is algorithm C.
\newpage

\item 
Collaborators: Mary Stufflebeam, Jacob Newman
\begin{enumerate}
\item $T (n) = 2T (n/3) + 1$
\begin{equation*}
\Theta(1)
\end{equation*}
\item $T (n) = 5T (n/4) + n$
\begin{equation*}
\Theta\left(n^{\log_4 5}\right)
\end{equation*}
\item $T (n) = 7T (n/7) + n$
\begin{equation*}
\Theta(n \log n)
\end{equation*}
\item $T (n) = 9T (n/3) + n^2$
\begin{equation*}
\Theta(n^2 \log n)
\end{equation*}
\item $T (n) = 8T (n/2) + n^3$
\begin{equation*}
\Theta(n^3 \log n)
\end{equation*}
\item $T (n) = 49T (n/25) + n^{3/2} \log n$

Since the additive side of the equation is much larger than $\log_{25}49$, then
it is dominated by that side, therefore, the growth is on the order of:
\begin{equation*}
\Theta(n^{3/2} \log n)
\end{equation*}
\item $T (n) = T (n - 1) + 2$
\begin{equation*}
\Theta(n)
\end{equation*}
\item $T (n) = T (n - 1) + n^c$, where $c \ge 1$ is a constant
\begin{equation*}
\Theta(n^{c+1})
\end{equation*}
\item $T (n) = T (n - 1) + c^n$, where $c > 1$ is some constant
\begin{equation*}
\Theta(c^n)
\end{equation*}
\item $T (n) = 2T (n - 1) + 1$
\begin{equation*}
\Theta(2^n)
\end{equation*}
\item $T (n) = T (\sqrt{n}) + 1$

For the $k$th recursion the value should be some constant c:
\begin{equation*}
n^{\frac{1}{2^k}} = c
\end{equation*}
When we solve for k we get:
\begin{eqnarray*}
\frac{1}{2^k} \log n &=& \log c\\
\frac{\log n}{\log c} &=& 2^k\\
\log \left( \frac{\log n}{\log c} \right) &=& k\\
\log (\log n) - \log(\log c) &=&  k
\end{eqnarray*}
Therefore the growth is:
\begin{equation*}
\Theta(\log (\log n))
\end{equation*}
\end{enumerate}

\newpage

\item 
Collaborators: Jacob Newman, Mary Stufflebeam

Sum:
\begin{eqnarray*}
\omega &=& e^{\frac{i2\pi}{n}}\\
\sum_{k=0}^{n-1} \omega^k &=& \frac{1 - \omega^n}{1 - \omega}\\
&=& 0
\end{eqnarray*}

Product (even):
You have the product of the following numbers:
\begin{equation*}
e^{0}, e^{\frac{i2\pi}{n}}, e^{\frac{i4\pi}{n}},
e^{\frac{i6\pi}{n}}, \dots, e^{\frac{i(n-1)\pi}{n}}
\end{equation*}
If you pair them up in the following way:
\begin{equation*}
\left(e^{\frac{i2\pi}{n}}\right)\left(e^{\frac{i(n-1)\pi}{n}}\right)
\end{equation*}

Once you pair them all up, each pair is $1$, with the left over $n/2$.
Therefore, the product is: $-1$

Product (odd):
The argument is the same as in the previous problem, except there is no $n/2$
case. So the final product is: $1$
\newpage

\item

Collaborators: Hayg Astorian, Mary Stufflebeam, Andre Crabb
\begin{enumerate}

\item

Pseudocode for the algorithm. 
\begin{verbatim}
def majority(list):
  if len(list) == 1:
    return list[0]
  candidate-l = majority(list[:len(list)/2])
  candidate-r = majority(list[len(list)/2:])
  c1 = 0  c2 = 0
  for item in list:
    if item == candidate-l:
      c1 += 1
    if item == candidate-r:
      c2 += 1
  if c1 > len(list)/2:
    return candidate-l
  if c2 > len(list)/2:
    return candidate-r
  return None
\end{verbatim}

The recurrence for this algorithm is:
\begin{equation*}
T(n) = 2T\left( \frac{n}{2} \right) + O(n) = O(n\log n)
\end{equation*}

\item

With a simple algorithm, we can perform this task in linear time using a
two-pass approach. 

The first pass is as follows:

The first element is considered to be a ``candidate'' for the correct majority.
For each element, if it is the same, we increment a count associated with this
candidate, and otherwise, we decrement it. Once the counter for a candidate
reaches zero, we replace it with the current number. 

Once we reach the end of this pass, the element that we have is our best guess
at a majority element, we perform the second pass where we add up the instances
of the final candidate. If the number of that element is more than half the size
of the list, then it is a majority element. Otherwise we have no majority
element. 

This algorithm is also commonly known as Moore's Voting Algorithm.

\end{enumerate}


\newpage

\item
Collaborators: Andre Crabb, Albert Eloyan, Hayg Astorian 

\begin{enumerate}

\item
If we have a square of size $d \times d$, then the only way that we can have
four points that are no more than $d$ distance away, then they must be on the
exact corners of the square. Any more points and it would be too close to one of
the other points and therefore $d$ would be smaller, which is a contradiction. 

Therefore, we have no way to have more than 4 points on any one side within a
$d \times d$ square. 

\item
If the minimum is in L or R, then the algorithm is clearly correct. 

In the case where there is a minimum in M, then things are a bit more
complicated. 

Using the property found in the previous step, if we go from the first element
and move in one direction, then at most the next 6 points could be exactly $d$
distance away from the initial point, therefore, if we check 7, then we will be
guaranteed to find a minimum if there is one. 

\item
\begin{verbatim}
def closest-pair(points):
  find an x value # O(n)
  pL, qL = closest-pair(L)
  pR, qR = closest-pair(R)
  find the d value
  sort the elements in the range x-d, x+d # O(nlogn)
  check the elements # O(n)
  return the minimum of the three sets of points
\end{verbatim}
Therefore, the recurrence is:
\begin{equation*}
T(n) = 2T(n/2) + O(n\log n)
\end{equation*}
Since we have the recurrence in the general form:
\begin{equation*}
T(n) = 2T(n/2) + f(n)
\end{equation*}
Where $f(n)$ is of the form:
\begin{equation*}
n^{\log_b a} \log^k n
\end{equation*}
Therefore, the value is:
\begin{equation*}
O(n^{\log_b a} \log^{k+1} n)
\end{equation*}
Therefore, the big O for this recurrence is:
\begin{equation*}
O(n\log^2n)
\end{equation*}
Sources: wikipedia
\item
Before running the algorithm, we sort the points by y value, and pass it in
to the algorithm. By carefully maintaining the order of the points, we never
have to sort the points again, thereby making all of the sub-calls run in linear
time, giving us the recurrence:
\begin{equation*}
T(n) = 2T\left( \frac{n}{2} \right) + O(n)
\end{equation*}
Which, by masters theorem is:
\begin{equation*}
O(n \log n)
\end{equation*}
And since the sort is done once in $n \log n$ time, then the entire algorithm is
also in $n \log n$ time. 
\end{enumerate}
\end{enumerate}
\end{document}
